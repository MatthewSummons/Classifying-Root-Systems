% In this file you should put the actual content of the blueprint.
% It will be used both by the web and the print version.
% It should *not* include the \begin{document}
%
% If you want to split the blueprint content into several files then
% the current file can be a simple sequence of \input. Otherwise It
% can start with a \section or \chapter for instance.


\chapter{Classifying Root Systems}
\section{Root Systems}

\begin{nameddefinition}{Root System}
    Let $E$ be a finite-dimensional Euclidean space with an inner product $\langle \cdot, \cdot \rangle$.
    
    A \textbf{root system} in $E$ is a tuple (E, $\Phi$), where $\Phi$ is a finite non-empty set of non-zero vectors (called roots) satisfying the following properties:
    \begin{enumerate}
        \item $\Phi$ spans $E$.
        \item For every root $\alpha \in \Phi$, the set $\Phi$ is closed under reflection through the hyperplane orthogonal to $\alpha$.
        That is, for any two roots $\alpha, \beta \in \Phi$, the set $\Phi$ contains the element
        \begin{equation*}
            \sigma_\alpha(\beta) = \beta - \frac{2 \langle \alpha, \beta \rangle}{\langle \alpha, \alpha \rangle} \alpha.
        \end{equation*}
    \end{enumerate}

    The rank of the root system is the dimension of the Euclidean space $E$.
\end{nameddefinition}

For convenience and in contexts where the inner product space is clear, the root system is often referred to simply as $\Phi$.

\begin{example}
    $R_0 = \{\pm \alpha\}$, where $\alpha$ is any fixed non-zero real number, constitutes a root system in $\mathbb{R}$.
\end{example}

\begin{nameddefinition}{Reduced Root System}
    If a root system satisfies the condition that the only multiples of a root, $\alpha$, that are in the root system are $\pm \alpha$, then the root system is said to be \textbf{reduced}.
\end{nameddefinition}

\begin{nameddefinition}{Crystallographic Root System}
    If a root system satisfies the integrality condition below, then it is said to be \textbf{crystallographic}.
    \begin{equation*}
        \langle \langle \beta, \alpha \rangle \rangle :=  \frac{2 \langle \alpha, \beta \rangle}{\langle \alpha, \alpha \rangle} \in \mathbb{Z} \quad \text{for all } \alpha, \beta \in \Phi.
    \end{equation*}
\end{nameddefinition}


Throughout this text, denote $e_i$ as the $i$-th standard basis vector in $\mathbb{R}^n$. Then, in combinations such as $\pm e_i \pm e_j$, the signs may be chosen independently.

\begin{example}
    The set $R_1$, shown below, is a root system in $\mathbb{R}^2$ that is neither reduced nor crystallographic.
    
    \begin{equation*}
        R_1 = \{
            \pm e_1, (\pm \frac{\sqrt{3}}{2}, \pm \frac{1}{2}), (\pm \sqrt{3}, \pm 1) 
        \} 
    \end{equation*}
    $R_1$ spans $\mathbb{R}^2$ and is closed under reflection through the hyperplane orthogonal to any root, hence it is a root system.
    
    However, is is not a \textbf{reduced} root system since a scalar multiple of an element in $R_1$, namely $2 \cdot (\pm \frac{1}{2}, \pm \frac{\sqrt{3}}{2})$, is contained in $R_1$ itself. 
    It is also not a \textbf{crystallographic} root system because $ \langle \langle e_1, (\pm \frac{\sqrt{3}}{2}, \pm \frac{1}{2}) \rangle \rangle = \frac{\sqrt{3}}{2} \notin \mathbb{Z}$.
\end{example}

\begin{example}
    If we remove the redundant multiple in $R_1$ above, we obtain a reduced, non-crystallographic root system $R_2$.
    
    \begin{equation*}
        R_2 = \{
            \pm e_1, (\pm \frac{\sqrt{3}}{2}, \pm \frac{1}{2})
        \} 
    \end{equation*}
\end{example}

One can also construct examples of non-reduced crystallographic root systems. Consider the following example,

\begin{example}
    The set $R_3$ is a root system in $\mathbb{R}^2$ that is crystallographic but not reduced.
    
    \begin{equation*}
        R_3 = \{
            \pm e_1, \pm e_2, \pm 2 e_1
        \} 
    \end{equation*}
    $R_3$ spans $\mathbb{R}^2$ and is closed under reflection through the hyperplane orthogonal to any root, hence it is a root system.
    It is a \textbf{crystallographic} root system because $\langle \langle k e_1, e_2 \rangle \rangle = 0$ and $\langle \langle k e_1, k' e_1 \rangle \rangle = kk' \in \mathbb{Z}$, where $k,k \in \{\pm 1, \pm 2\}$.
    However, is is not a \textbf{reduced} root system since $2 e_1 \in R_3$.
\end{example}


\begin{example}
    The set $R_4$ is a root system in $\mathbb{R}^2$ that is reduced and crystallographic.
    
    \begin{equation*}
        R_4 = \{
            \pm e_1, \pm e_2
            \} 
        \end{equation*}
    \end{example}

Therefore, we see that a root system may be reduced, crystallographic, both, or neither.
This report concerns itself primarily with reduced, crystallographic root systems, simply referred to as root systems henceforth, unless otherwise specified. \newline

Since we aim to classify all root systems, upto isomorhism, it is important to understand when two root systems are isomorphic. \newline

\begin{nameddefinition}{Root System Isomorphism}
    Two root systems $(E, \Phi)$ and $(F,\Psi)$ are said to be isomorphic if there exists a linear isomorphism $\varphi: E \to F$ such that $\varphi(\Phi) = \Psi$
    and preserves the number $\langle \langle x, y \rangle \rangle$ for each pair of roots.
\end{nameddefinition}

\begin{example}
    The root systems $R = \{ \pm e_1, \pm e_2 \} $ and $R' = \{ \pm e_1 \pm e_2 \}$ are isomorphic.
\end{example}

\begin{example}
    The root systems $S = \{ \pm e_1, \pm e_2, \pm e_1 \pm e_2 \} $ and $S' = \{ \pm e_1, (\pm \frac{\sqrt{3}}{2}, \pm \frac{1}{2}) \}$ are not isomorphic.
\end{example}

\section{Classifying Root Systems of Small Rank}

% Classification of Rank 1 Root Systems
\begin{namedtheorem}{Classification of Rank 1 Root Systems}
    The only reduced, crystallographic root systems of rank 1 are $R_0 = \{ \pm \alpha \}$, where $\alpha$ is any fixed non-zero real number.
\end{namedtheorem}

\begin{proof}
    Let $\Phi$ be a reduced, crystallographic root system of rank 1.
    Let $\alpha \in \Phi$ be a non-zero root.
    Since $\Phi$ is reduced, the only multiples of $\alpha$ in $\Phi$ are $\pm \alpha$.
    Therefore, $\Phi = \{ \pm \alpha \}$.
\end{proof}

